
% Default to the notebook output style

    


% Inherit from the specified cell style.




    
\documentclass[11pt]{article}

    
    
    \usepackage[T1]{fontenc}
    % Nicer default font (+ math font) than Computer Modern for most use cases
    \usepackage{mathpazo}

    % Basic figure setup, for now with no caption control since it's done
    % automatically by Pandoc (which extracts ![](path) syntax from Markdown).
    \usepackage{graphicx}
    % We will generate all images so they have a width \maxwidth. This means
    % that they will get their normal width if they fit onto the page, but
    % are scaled down if they would overflow the margins.
    \makeatletter
    \def\maxwidth{\ifdim\Gin@nat@width>\linewidth\linewidth
    \else\Gin@nat@width\fi}
    \makeatother
    \let\Oldincludegraphics\includegraphics
    % Set max figure width to be 80% of text width, for now hardcoded.
    \renewcommand{\includegraphics}[1]{\Oldincludegraphics[width=.8\maxwidth]{#1}}
    % Ensure that by default, figures have no caption (until we provide a
    % proper Figure object with a Caption API and a way to capture that
    % in the conversion process - todo).
    \usepackage{caption}
    \DeclareCaptionLabelFormat{nolabel}{}
    \captionsetup{labelformat=nolabel}

    \usepackage{adjustbox} % Used to constrain images to a maximum size 
    \usepackage{xcolor} % Allow colors to be defined
    \usepackage{enumerate} % Needed for markdown enumerations to work
    \usepackage{geometry} % Used to adjust the document margins
    \usepackage{amsmath} % Equations
    \usepackage{amssymb} % Equations
    \usepackage{textcomp} % defines textquotesingle
    % Hack from http://tex.stackexchange.com/a/47451/13684:
    \AtBeginDocument{%
        \def\PYZsq{\textquotesingle}% Upright quotes in Pygmentized code
    }
    \usepackage{upquote} % Upright quotes for verbatim code
    \usepackage{eurosym} % defines \euro
    \usepackage[mathletters]{ucs} % Extended unicode (utf-8) support
    \usepackage[utf8x]{inputenc} % Allow utf-8 characters in the tex document
    \usepackage{fancyvrb} % verbatim replacement that allows latex
    \usepackage{grffile} % extends the file name processing of package graphics 
                         % to support a larger range 
    % The hyperref package gives us a pdf with properly built
    % internal navigation ('pdf bookmarks' for the table of contents,
    % internal cross-reference links, web links for URLs, etc.)
    \usepackage{hyperref}
    \usepackage{longtable} % longtable support required by pandoc >1.10
    \usepackage{booktabs}  % table support for pandoc > 1.12.2
    \usepackage[inline]{enumitem} % IRkernel/repr support (it uses the enumerate* environment)
    \usepackage[normalem]{ulem} % ulem is needed to support strikethroughs (\sout)
                                % normalem makes italics be italics, not underlines
    

    
    
    % Colors for the hyperref package
    \definecolor{urlcolor}{rgb}{0,.145,.698}
    \definecolor{linkcolor}{rgb}{.71,0.21,0.01}
    \definecolor{citecolor}{rgb}{.12,.54,.11}

    % ANSI colors
    \definecolor{ansi-black}{HTML}{3E424D}
    \definecolor{ansi-black-intense}{HTML}{282C36}
    \definecolor{ansi-red}{HTML}{E75C58}
    \definecolor{ansi-red-intense}{HTML}{B22B31}
    \definecolor{ansi-green}{HTML}{00A250}
    \definecolor{ansi-green-intense}{HTML}{007427}
    \definecolor{ansi-yellow}{HTML}{DDB62B}
    \definecolor{ansi-yellow-intense}{HTML}{B27D12}
    \definecolor{ansi-blue}{HTML}{208FFB}
    \definecolor{ansi-blue-intense}{HTML}{0065CA}
    \definecolor{ansi-magenta}{HTML}{D160C4}
    \definecolor{ansi-magenta-intense}{HTML}{A03196}
    \definecolor{ansi-cyan}{HTML}{60C6C8}
    \definecolor{ansi-cyan-intense}{HTML}{258F8F}
    \definecolor{ansi-white}{HTML}{C5C1B4}
    \definecolor{ansi-white-intense}{HTML}{A1A6B2}

    % commands and environments needed by pandoc snippets
    % extracted from the output of `pandoc -s`
    \providecommand{\tightlist}{%
      \setlength{\itemsep}{0pt}\setlength{\parskip}{0pt}}
    \DefineVerbatimEnvironment{Highlighting}{Verbatim}{commandchars=\\\{\}}
    % Add ',fontsize=\small' for more characters per line
    \newenvironment{Shaded}{}{}
    \newcommand{\KeywordTok}[1]{\textcolor[rgb]{0.00,0.44,0.13}{\textbf{{#1}}}}
    \newcommand{\DataTypeTok}[1]{\textcolor[rgb]{0.56,0.13,0.00}{{#1}}}
    \newcommand{\DecValTok}[1]{\textcolor[rgb]{0.25,0.63,0.44}{{#1}}}
    \newcommand{\BaseNTok}[1]{\textcolor[rgb]{0.25,0.63,0.44}{{#1}}}
    \newcommand{\FloatTok}[1]{\textcolor[rgb]{0.25,0.63,0.44}{{#1}}}
    \newcommand{\CharTok}[1]{\textcolor[rgb]{0.25,0.44,0.63}{{#1}}}
    \newcommand{\StringTok}[1]{\textcolor[rgb]{0.25,0.44,0.63}{{#1}}}
    \newcommand{\CommentTok}[1]{\textcolor[rgb]{0.38,0.63,0.69}{\textit{{#1}}}}
    \newcommand{\OtherTok}[1]{\textcolor[rgb]{0.00,0.44,0.13}{{#1}}}
    \newcommand{\AlertTok}[1]{\textcolor[rgb]{1.00,0.00,0.00}{\textbf{{#1}}}}
    \newcommand{\FunctionTok}[1]{\textcolor[rgb]{0.02,0.16,0.49}{{#1}}}
    \newcommand{\RegionMarkerTok}[1]{{#1}}
    \newcommand{\ErrorTok}[1]{\textcolor[rgb]{1.00,0.00,0.00}{\textbf{{#1}}}}
    \newcommand{\NormalTok}[1]{{#1}}
    
    % Additional commands for more recent versions of Pandoc
    \newcommand{\ConstantTok}[1]{\textcolor[rgb]{0.53,0.00,0.00}{{#1}}}
    \newcommand{\SpecialCharTok}[1]{\textcolor[rgb]{0.25,0.44,0.63}{{#1}}}
    \newcommand{\VerbatimStringTok}[1]{\textcolor[rgb]{0.25,0.44,0.63}{{#1}}}
    \newcommand{\SpecialStringTok}[1]{\textcolor[rgb]{0.73,0.40,0.53}{{#1}}}
    \newcommand{\ImportTok}[1]{{#1}}
    \newcommand{\DocumentationTok}[1]{\textcolor[rgb]{0.73,0.13,0.13}{\textit{{#1}}}}
    \newcommand{\AnnotationTok}[1]{\textcolor[rgb]{0.38,0.63,0.69}{\textbf{\textit{{#1}}}}}
    \newcommand{\CommentVarTok}[1]{\textcolor[rgb]{0.38,0.63,0.69}{\textbf{\textit{{#1}}}}}
    \newcommand{\VariableTok}[1]{\textcolor[rgb]{0.10,0.09,0.49}{{#1}}}
    \newcommand{\ControlFlowTok}[1]{\textcolor[rgb]{0.00,0.44,0.13}{\textbf{{#1}}}}
    \newcommand{\OperatorTok}[1]{\textcolor[rgb]{0.40,0.40,0.40}{{#1}}}
    \newcommand{\BuiltInTok}[1]{{#1}}
    \newcommand{\ExtensionTok}[1]{{#1}}
    \newcommand{\PreprocessorTok}[1]{\textcolor[rgb]{0.74,0.48,0.00}{{#1}}}
    \newcommand{\AttributeTok}[1]{\textcolor[rgb]{0.49,0.56,0.16}{{#1}}}
    \newcommand{\InformationTok}[1]{\textcolor[rgb]{0.38,0.63,0.69}{\textbf{\textit{{#1}}}}}
    \newcommand{\WarningTok}[1]{\textcolor[rgb]{0.38,0.63,0.69}{\textbf{\textit{{#1}}}}}
    
    
    % Define a nice break command that doesn't care if a line doesn't already
    % exist.
    \def\br{\hspace*{\fill} \\* }
    % Math Jax compatability definitions
    \def\gt{>}
    \def\lt{<}
    % Document parameters
    \title{ Bayesian Data Analysis - Assignment 8}
    
    
    

    % Pygments definitions
    
\makeatletter
\def\PY@reset{\let\PY@it=\relax \let\PY@bf=\relax%
    \let\PY@ul=\relax \let\PY@tc=\relax%
    \let\PY@bc=\relax \let\PY@ff=\relax}
\def\PY@tok#1{\csname PY@tok@#1\endcsname}
\def\PY@toks#1+{\ifx\relax#1\empty\else%
    \PY@tok{#1}\expandafter\PY@toks\fi}
\def\PY@do#1{\PY@bc{\PY@tc{\PY@ul{%
    \PY@it{\PY@bf{\PY@ff{#1}}}}}}}
\def\PY#1#2{\PY@reset\PY@toks#1+\relax+\PY@do{#2}}

\expandafter\def\csname PY@tok@gd\endcsname{\def\PY@tc##1{\textcolor[rgb]{0.63,0.00,0.00}{##1}}}
\expandafter\def\csname PY@tok@gu\endcsname{\let\PY@bf=\textbf\def\PY@tc##1{\textcolor[rgb]{0.50,0.00,0.50}{##1}}}
\expandafter\def\csname PY@tok@gt\endcsname{\def\PY@tc##1{\textcolor[rgb]{0.00,0.27,0.87}{##1}}}
\expandafter\def\csname PY@tok@gs\endcsname{\let\PY@bf=\textbf}
\expandafter\def\csname PY@tok@gr\endcsname{\def\PY@tc##1{\textcolor[rgb]{1.00,0.00,0.00}{##1}}}
\expandafter\def\csname PY@tok@cm\endcsname{\let\PY@it=\textit\def\PY@tc##1{\textcolor[rgb]{0.25,0.50,0.50}{##1}}}
\expandafter\def\csname PY@tok@vg\endcsname{\def\PY@tc##1{\textcolor[rgb]{0.10,0.09,0.49}{##1}}}
\expandafter\def\csname PY@tok@vi\endcsname{\def\PY@tc##1{\textcolor[rgb]{0.10,0.09,0.49}{##1}}}
\expandafter\def\csname PY@tok@vm\endcsname{\def\PY@tc##1{\textcolor[rgb]{0.10,0.09,0.49}{##1}}}
\expandafter\def\csname PY@tok@mh\endcsname{\def\PY@tc##1{\textcolor[rgb]{0.40,0.40,0.40}{##1}}}
\expandafter\def\csname PY@tok@cs\endcsname{\let\PY@it=\textit\def\PY@tc##1{\textcolor[rgb]{0.25,0.50,0.50}{##1}}}
\expandafter\def\csname PY@tok@ge\endcsname{\let\PY@it=\textit}
\expandafter\def\csname PY@tok@vc\endcsname{\def\PY@tc##1{\textcolor[rgb]{0.10,0.09,0.49}{##1}}}
\expandafter\def\csname PY@tok@il\endcsname{\def\PY@tc##1{\textcolor[rgb]{0.40,0.40,0.40}{##1}}}
\expandafter\def\csname PY@tok@go\endcsname{\def\PY@tc##1{\textcolor[rgb]{0.53,0.53,0.53}{##1}}}
\expandafter\def\csname PY@tok@cp\endcsname{\def\PY@tc##1{\textcolor[rgb]{0.74,0.48,0.00}{##1}}}
\expandafter\def\csname PY@tok@gi\endcsname{\def\PY@tc##1{\textcolor[rgb]{0.00,0.63,0.00}{##1}}}
\expandafter\def\csname PY@tok@gh\endcsname{\let\PY@bf=\textbf\def\PY@tc##1{\textcolor[rgb]{0.00,0.00,0.50}{##1}}}
\expandafter\def\csname PY@tok@ni\endcsname{\let\PY@bf=\textbf\def\PY@tc##1{\textcolor[rgb]{0.60,0.60,0.60}{##1}}}
\expandafter\def\csname PY@tok@nl\endcsname{\def\PY@tc##1{\textcolor[rgb]{0.63,0.63,0.00}{##1}}}
\expandafter\def\csname PY@tok@nn\endcsname{\let\PY@bf=\textbf\def\PY@tc##1{\textcolor[rgb]{0.00,0.00,1.00}{##1}}}
\expandafter\def\csname PY@tok@no\endcsname{\def\PY@tc##1{\textcolor[rgb]{0.53,0.00,0.00}{##1}}}
\expandafter\def\csname PY@tok@na\endcsname{\def\PY@tc##1{\textcolor[rgb]{0.49,0.56,0.16}{##1}}}
\expandafter\def\csname PY@tok@nb\endcsname{\def\PY@tc##1{\textcolor[rgb]{0.00,0.50,0.00}{##1}}}
\expandafter\def\csname PY@tok@nc\endcsname{\let\PY@bf=\textbf\def\PY@tc##1{\textcolor[rgb]{0.00,0.00,1.00}{##1}}}
\expandafter\def\csname PY@tok@nd\endcsname{\def\PY@tc##1{\textcolor[rgb]{0.67,0.13,1.00}{##1}}}
\expandafter\def\csname PY@tok@ne\endcsname{\let\PY@bf=\textbf\def\PY@tc##1{\textcolor[rgb]{0.82,0.25,0.23}{##1}}}
\expandafter\def\csname PY@tok@nf\endcsname{\def\PY@tc##1{\textcolor[rgb]{0.00,0.00,1.00}{##1}}}
\expandafter\def\csname PY@tok@si\endcsname{\let\PY@bf=\textbf\def\PY@tc##1{\textcolor[rgb]{0.73,0.40,0.53}{##1}}}
\expandafter\def\csname PY@tok@s2\endcsname{\def\PY@tc##1{\textcolor[rgb]{0.73,0.13,0.13}{##1}}}
\expandafter\def\csname PY@tok@nt\endcsname{\let\PY@bf=\textbf\def\PY@tc##1{\textcolor[rgb]{0.00,0.50,0.00}{##1}}}
\expandafter\def\csname PY@tok@nv\endcsname{\def\PY@tc##1{\textcolor[rgb]{0.10,0.09,0.49}{##1}}}
\expandafter\def\csname PY@tok@s1\endcsname{\def\PY@tc##1{\textcolor[rgb]{0.73,0.13,0.13}{##1}}}
\expandafter\def\csname PY@tok@dl\endcsname{\def\PY@tc##1{\textcolor[rgb]{0.73,0.13,0.13}{##1}}}
\expandafter\def\csname PY@tok@ch\endcsname{\let\PY@it=\textit\def\PY@tc##1{\textcolor[rgb]{0.25,0.50,0.50}{##1}}}
\expandafter\def\csname PY@tok@m\endcsname{\def\PY@tc##1{\textcolor[rgb]{0.40,0.40,0.40}{##1}}}
\expandafter\def\csname PY@tok@gp\endcsname{\let\PY@bf=\textbf\def\PY@tc##1{\textcolor[rgb]{0.00,0.00,0.50}{##1}}}
\expandafter\def\csname PY@tok@sh\endcsname{\def\PY@tc##1{\textcolor[rgb]{0.73,0.13,0.13}{##1}}}
\expandafter\def\csname PY@tok@ow\endcsname{\let\PY@bf=\textbf\def\PY@tc##1{\textcolor[rgb]{0.67,0.13,1.00}{##1}}}
\expandafter\def\csname PY@tok@sx\endcsname{\def\PY@tc##1{\textcolor[rgb]{0.00,0.50,0.00}{##1}}}
\expandafter\def\csname PY@tok@bp\endcsname{\def\PY@tc##1{\textcolor[rgb]{0.00,0.50,0.00}{##1}}}
\expandafter\def\csname PY@tok@c1\endcsname{\let\PY@it=\textit\def\PY@tc##1{\textcolor[rgb]{0.25,0.50,0.50}{##1}}}
\expandafter\def\csname PY@tok@fm\endcsname{\def\PY@tc##1{\textcolor[rgb]{0.00,0.00,1.00}{##1}}}
\expandafter\def\csname PY@tok@o\endcsname{\def\PY@tc##1{\textcolor[rgb]{0.40,0.40,0.40}{##1}}}
\expandafter\def\csname PY@tok@kc\endcsname{\let\PY@bf=\textbf\def\PY@tc##1{\textcolor[rgb]{0.00,0.50,0.00}{##1}}}
\expandafter\def\csname PY@tok@c\endcsname{\let\PY@it=\textit\def\PY@tc##1{\textcolor[rgb]{0.25,0.50,0.50}{##1}}}
\expandafter\def\csname PY@tok@mf\endcsname{\def\PY@tc##1{\textcolor[rgb]{0.40,0.40,0.40}{##1}}}
\expandafter\def\csname PY@tok@err\endcsname{\def\PY@bc##1{\setlength{\fboxsep}{0pt}\fcolorbox[rgb]{1.00,0.00,0.00}{1,1,1}{\strut ##1}}}
\expandafter\def\csname PY@tok@mb\endcsname{\def\PY@tc##1{\textcolor[rgb]{0.40,0.40,0.40}{##1}}}
\expandafter\def\csname PY@tok@ss\endcsname{\def\PY@tc##1{\textcolor[rgb]{0.10,0.09,0.49}{##1}}}
\expandafter\def\csname PY@tok@sr\endcsname{\def\PY@tc##1{\textcolor[rgb]{0.73,0.40,0.53}{##1}}}
\expandafter\def\csname PY@tok@mo\endcsname{\def\PY@tc##1{\textcolor[rgb]{0.40,0.40,0.40}{##1}}}
\expandafter\def\csname PY@tok@kd\endcsname{\let\PY@bf=\textbf\def\PY@tc##1{\textcolor[rgb]{0.00,0.50,0.00}{##1}}}
\expandafter\def\csname PY@tok@mi\endcsname{\def\PY@tc##1{\textcolor[rgb]{0.40,0.40,0.40}{##1}}}
\expandafter\def\csname PY@tok@kn\endcsname{\let\PY@bf=\textbf\def\PY@tc##1{\textcolor[rgb]{0.00,0.50,0.00}{##1}}}
\expandafter\def\csname PY@tok@cpf\endcsname{\let\PY@it=\textit\def\PY@tc##1{\textcolor[rgb]{0.25,0.50,0.50}{##1}}}
\expandafter\def\csname PY@tok@kr\endcsname{\let\PY@bf=\textbf\def\PY@tc##1{\textcolor[rgb]{0.00,0.50,0.00}{##1}}}
\expandafter\def\csname PY@tok@s\endcsname{\def\PY@tc##1{\textcolor[rgb]{0.73,0.13,0.13}{##1}}}
\expandafter\def\csname PY@tok@kp\endcsname{\def\PY@tc##1{\textcolor[rgb]{0.00,0.50,0.00}{##1}}}
\expandafter\def\csname PY@tok@w\endcsname{\def\PY@tc##1{\textcolor[rgb]{0.73,0.73,0.73}{##1}}}
\expandafter\def\csname PY@tok@kt\endcsname{\def\PY@tc##1{\textcolor[rgb]{0.69,0.00,0.25}{##1}}}
\expandafter\def\csname PY@tok@sc\endcsname{\def\PY@tc##1{\textcolor[rgb]{0.73,0.13,0.13}{##1}}}
\expandafter\def\csname PY@tok@sb\endcsname{\def\PY@tc##1{\textcolor[rgb]{0.73,0.13,0.13}{##1}}}
\expandafter\def\csname PY@tok@sa\endcsname{\def\PY@tc##1{\textcolor[rgb]{0.73,0.13,0.13}{##1}}}
\expandafter\def\csname PY@tok@k\endcsname{\let\PY@bf=\textbf\def\PY@tc##1{\textcolor[rgb]{0.00,0.50,0.00}{##1}}}
\expandafter\def\csname PY@tok@se\endcsname{\let\PY@bf=\textbf\def\PY@tc##1{\textcolor[rgb]{0.73,0.40,0.13}{##1}}}
\expandafter\def\csname PY@tok@sd\endcsname{\let\PY@it=\textit\def\PY@tc##1{\textcolor[rgb]{0.73,0.13,0.13}{##1}}}

\def\PYZbs{\char`\\}
\def\PYZus{\char`\_}
\def\PYZob{\char`\{}
\def\PYZcb{\char`\}}
\def\PYZca{\char`\^}
\def\PYZam{\char`\&}
\def\PYZlt{\char`\<}
\def\PYZgt{\char`\>}
\def\PYZsh{\char`\#}
\def\PYZpc{\char`\%}
\def\PYZdl{\char`\$}
\def\PYZhy{\char`\-}
\def\PYZsq{\char`\'}
\def\PYZdq{\char`\"}
\def\PYZti{\char`\~}
% for compatibility with earlier versions
\def\PYZat{@}
\def\PYZlb{[}
\def\PYZrb{]}
\makeatother


    % Exact colors from NB
    \definecolor{incolor}{rgb}{0.0, 0.0, 0.5}
    \definecolor{outcolor}{rgb}{0.545, 0.0, 0.0}



    
    % Prevent overflowing lines due to hard-to-break entities
    \sloppy 
    % Setup hyperref package
    \hypersetup{
      breaklinks=true,  % so long urls are correctly broken across lines
      colorlinks=true,
      urlcolor=urlcolor,
      linkcolor=linkcolor,
      citecolor=citecolor,
      }
    % Slightly bigger margins than the latex defaults
    
    \geometry{verbose,tmargin=1in,bmargin=1in,lmargin=1in,rmargin=1in}
    
    

    \begin{document}
    
    
    \maketitle
    
    

    
    \hypertarget{model-assessment-loo-cv-for-factory-data-with-stan}{%
\section{Model assessment: LOO-CV for factory data with
Stan}\label{model-assessment-loo-cv-for-factory-data-with-stan}}

    If this is to be executed, please execute the source code at the end of
the report first. The report addresses the first two points of the
assignment for each model, then give conlusion on the differences
between models.

    \hypertarget{separate-model}{%
\subsection{SEPARATE MODEL}\label{separate-model}}

    \textbf{psis-loo, p\_eff, k-values}

    \begin{Verbatim}[commandchars=\\\{\}]
{\color{incolor}In [{\color{incolor}56}]:} \PY{n}{get\PYZus{}separate\PYZus{}result}\PY{p}{(}\PY{p}{)}
\end{Verbatim}


    \begin{Verbatim}[commandchars=\\\{\}]
PSIS-LOO value is -132.1610216250452
p\_eff value is 9.689751572428591
Histogram of k-values:

    \end{Verbatim}

    \begin{center}
    \adjustimage{max size={0.9\linewidth}{0.9\paperheight}}{output_4_1.png}
    \end{center}
    { \hspace*{\fill} \\}
    
    \textbf{Assement based on k-values}

As can be seen from the histogram, there are \textbf{several k-values}
\textgreater{} 0.7 , which is still very small w.r.t the number of
samples. Then PSIS-LOO is considered \textbf{reliable} in this case

    \hypertarget{pool-model}{%
\subsection{POOL MODEL}\label{pool-model}}

    \textbf{psis-loo, p\_eff, k-values}

    \begin{Verbatim}[commandchars=\\\{\}]
{\color{incolor}In [{\color{incolor}54}]:} \PY{n}{get\PYZus{}pool\PYZus{}result}\PY{p}{(}\PY{p}{)}
\end{Verbatim}


    \begin{Verbatim}[commandchars=\\\{\}]
PSIS-LOO value is -130.95795612034993
p\_eff value is 1.9928311101955671
Histogram of k-values:

    \end{Verbatim}

    \begin{center}
    \adjustimage{max size={0.9\linewidth}{0.9\paperheight}}{output_8_1.png}
    \end{center}
    { \hspace*{\fill} \\}
    
    \textbf{Assement based on k-values}

As can be seen from the histogram, \textbf{all k-values \textless{} 0.7}
(in fact all k-values are less than \textasciitilde{} 0.5). Then
PSIS-LOO is considered \textbf{very reliable} in this case

    \hypertarget{hierarchical-model}{%
\subsection{HIERARCHICAL MODEL}\label{hierarchical-model}}

    \textbf{psis-loo, p\_eff, k-values}

    \begin{Verbatim}[commandchars=\\\{\}]
{\color{incolor}In [{\color{incolor}55}]:} \PY{n}{get\PYZus{}hierarchical\PYZus{}result}\PY{p}{(}\PY{p}{)}
\end{Verbatim}


    \begin{Verbatim}[commandchars=\\\{\}]
PSIS-LOO value is -126.80815188646743
p\_eff value is 5.604461670553292
Histogram of k-values:

    \end{Verbatim}

    \begin{center}
    \adjustimage{max size={0.9\linewidth}{0.9\paperheight}}{output_12_1.png}
    \end{center}
    { \hspace*{\fill} \\}
    
    \textbf{Assement based on k-values}

From the histogram, \textbf{all k-values \textless{} 0.7}. Then PSIS-LOO
is considered \textbf{reliable} in this case

    \begin{center}\rule{0.5\linewidth}{\linethickness}\end{center}

\hypertarget{conclusion-on-difference-between-models}{%
\subsection{CONCLUSION ON DIFFERENCE BETWEEN
MODELS}\label{conclusion-on-difference-between-models}}

All models are reliable even with different distribution of k-values. If
we consider towards model with best predictive accuracy, then
\textbf{HIERARCHICAL MODEL} should be selected, because its PSIS-LOO
value is the highest (or the sum of log predictive density is the
highest)

    \begin{center}\rule{0.5\linewidth}{\linethickness}\end{center}

\textbf{SOURCE CODE}

    Some bootstraping code and common function

    \begin{Verbatim}[commandchars=\\\{\}]
{\color{incolor}In [{\color{incolor}52}]:} \PY{k+kn}{import} \PY{n+nn}{pystan}
         \PY{k+kn}{import} \PY{n+nn}{numpy} \PY{k+kn}{as} \PY{n+nn}{np}
         \PY{k+kn}{import} \PY{n+nn}{pickle}
         \PY{k+kn}{import} \PY{n+nn}{matplotlib.pyplot} \PY{k+kn}{as} \PY{n+nn}{plt}
         \PY{k+kn}{from} \PY{n+nn}{psis} \PY{k+kn}{import} \PY{n}{psisloo}
         
         \PY{n}{np}\PY{o}{.}\PY{n}{random}\PY{o}{.}\PY{n}{seed}\PY{p}{(}\PY{n}{seed}\PY{o}{=}\PY{l+m+mi}{123}\PY{p}{)}
         \PY{n}{raw\PYZus{}data} \PY{o}{=} \PY{n}{np}\PY{o}{.}\PY{n}{loadtxt}\PY{p}{(}\PY{l+s+s1}{\PYZsq{}}\PY{l+s+s1}{factory.txt}\PY{l+s+s1}{\PYZsq{}}\PY{p}{)}
         \PY{n}{data} \PY{o}{=} \PY{n+nb}{dict}\PY{p}{(}\PY{n}{N}\PY{o}{=}\PY{l+m+mi}{5}\PY{p}{,} \PY{n}{K}\PY{o}{=}\PY{l+m+mi}{6}\PY{p}{,} \PY{n}{y}\PY{o}{=}\PY{n}{raw\PYZus{}data}\PY{p}{)}
         
         \PY{k}{def} \PY{n+nf}{get\PYZus{}p\PYZus{}eff}\PY{p}{(}\PY{n}{log\PYZus{}lik}\PY{p}{,} \PY{n}{lppd\PYZus{}loocv}\PY{p}{)}\PY{p}{:}    
             \PY{n}{likelihoods} \PY{o}{=} \PY{n}{np}\PY{o}{.}\PY{n}{asarray}\PY{p}{(}\PY{p}{[}\PY{n}{np}\PY{o}{.}\PY{n}{exp}\PY{p}{(}\PY{n}{log\PYZus{}likelihood}\PY{o}{.}\PY{n}{flatten}\PY{p}{(}\PY{p}{)}\PY{p}{)} \PY{k}{for} \PY{n}{log\PYZus{}likelihood} \PY{o+ow}{in} \PY{n}{log\PYZus{}lik}\PY{p}{]}\PY{p}{)}
             \PY{n}{num\PYZus{}sim}\PY{p}{,} \PY{n}{num\PYZus{}obs} \PY{o}{=} \PY{n}{likelihoods}\PY{o}{.}\PY{n}{shape}
             \PY{n}{lppd} \PY{o}{=} \PY{l+m+mi}{0}
             \PY{k}{for} \PY{n}{obs} \PY{o+ow}{in} \PY{n+nb}{range}\PY{p}{(}\PY{n}{num\PYZus{}obs}\PY{p}{)}\PY{p}{:}
                 \PY{n}{lppd} \PY{o}{+}\PY{o}{=} \PY{n}{np}\PY{o}{.}\PY{n}{log}\PY{p}{(}\PY{n}{np}\PY{o}{.}\PY{n}{sum}\PY{p}{(}\PY{n}{likelihoods}\PY{p}{[}\PY{p}{:}\PY{p}{,} \PY{n}{obs}\PY{p}{]}\PY{p}{)} \PY{o}{/} \PY{n}{num\PYZus{}sim}\PY{p}{)}
             
             \PY{n}{p\PYZus{}eff} \PY{o}{=} \PY{n}{lppd} \PY{o}{\PYZhy{}} \PY{n}{lppd\PYZus{}loocv}
             \PY{k}{return} \PY{n}{p\PYZus{}eff}
             
         \PY{k}{def} \PY{n+nf}{extract\PYZus{}result\PYZus{}from\PYZus{}fit}\PY{p}{(}\PY{n}{samples}\PY{p}{,} \PY{n}{plot\PYZus{}title}\PY{o}{=}\PY{l+s+s1}{\PYZsq{}}\PY{l+s+s1}{\PYZsq{}}\PY{p}{)}\PY{p}{:}
             \PY{n}{log\PYZus{}lik\PYZus{}matrix} \PY{o}{=} \PY{n}{np}\PY{o}{.}\PY{n}{asarray}\PY{p}{(}\PY{p}{[}\PY{n}{single\PYZus{}sample}\PY{o}{.}\PY{n}{flatten}\PY{p}{(}\PY{p}{)} \PY{k}{for} \PY{n}{single\PYZus{}sample} \PY{o+ow}{in} \PY{n}{samples}\PY{p}{[}\PY{l+s+s1}{\PYZsq{}}\PY{l+s+s1}{log\PYZus{}lik}\PY{l+s+s1}{\PYZsq{}}\PY{p}{]}\PY{p}{]}\PY{p}{)}
             \PY{n}{loo}\PY{p}{,} \PY{n}{loos}\PY{p}{,} \PY{n}{ks} \PY{o}{=} \PY{n}{psisloo}\PY{p}{(}\PY{n}{log\PYZus{}lik\PYZus{}matrix}\PY{p}{)}
         
             \PY{c+c1}{\PYZsh{} Calculate p\PYZus{}eff}
             \PY{n}{p\PYZus{}eff} \PY{o}{=} \PY{n}{get\PYZus{}p\PYZus{}eff}\PY{p}{(}\PY{n}{log\PYZus{}lik\PYZus{}matrix}\PY{p}{,} \PY{n}{loo}\PY{p}{)}
         
             \PY{k}{print}\PY{p}{(}\PY{n}{f}\PY{l+s+s1}{\PYZsq{}}\PY{l+s+s1}{PSIS\PYZhy{}LOO value is \PYZob{}loo\PYZcb{}}\PY{l+s+s1}{\PYZsq{}}\PY{p}{)}
             \PY{k}{print}\PY{p}{(}\PY{n}{f}\PY{l+s+s1}{\PYZsq{}}\PY{l+s+s1}{p\PYZus{}eff value is \PYZob{}p\PYZus{}eff\PYZcb{}}\PY{l+s+s1}{\PYZsq{}}\PY{p}{)}
             \PY{k}{print}\PY{p}{(}\PY{l+s+s1}{\PYZsq{}}\PY{l+s+s1}{Histogram of k\PYZhy{}values:}\PY{l+s+s1}{\PYZsq{}}\PY{p}{)}
             \PY{n}{plt}\PY{o}{.}\PY{n}{hist}\PY{p}{(}\PY{n}{ks}\PY{p}{,} \PY{l+m+mi}{30}\PY{p}{)}
             \PY{n}{plt}\PY{o}{.}\PY{n}{title}\PY{p}{(}\PY{n}{plot\PYZus{}title}\PY{p}{)}
             \PY{n}{plt}\PY{o}{.}\PY{n}{show}\PY{p}{(}\PY{p}{)}
             
             
\end{Verbatim}


    Model codes. From top to bottom: separate, pool, hierarchical

    \begin{Verbatim}[commandchars=\\\{\}]
{\color{incolor}In [{\color{incolor}48}]:} \PY{o}{\PYZpc{}\PYZpc{}}\PY{k}{capture}
         separate\PYZus{}model = \PYZdq{}\PYZdq{}\PYZdq{}
         data \PYZob{}
           int\PYZlt{}lower=0\PYZgt{} N; // number of observations per machine
           int\PYZlt{}lower=0\PYZgt{} K; // number of machines
           matrix[N,K] y; // N * K matrix of observation
         \PYZcb{}
         parameters \PYZob{}
           real theta[K]; // group means
           real\PYZlt{}lower=0\PYZgt{} sigma[K]; // group std
         \PYZcb{}
         model \PYZob{}
           for (k in 1:K)
             y[:,k] \PYZti{} normal(theta[k], sigma[k]);
         \PYZcb{}
         generated quantities \PYZob{}
             matrix[N,K] log\PYZus{}lik;
             for (k in 1:K)
               for (i in 1:N)
                 log\PYZus{}lik[i,k] = normal\PYZus{}lpdf(y[i,k] | theta[k], sigma[k]);
         \PYZcb{}
         \PYZdq{}\PYZdq{}\PYZdq{}
         
         separate\PYZus{}sm = pystan.StanModel(model\PYZus{}code=separate\PYZus{}model)
         separate\PYZus{}fit = separate\PYZus{}sm.sampling(data=data, iter=2000, chains=4)
         
         pool\PYZus{}model = \PYZdq{}\PYZdq{}\PYZdq{}
         data \PYZob{}
           int\PYZlt{}lower=0\PYZgt{} N;
           int\PYZlt{}lower=0\PYZgt{} K;
           matrix[N,K] y;
         \PYZcb{}
         parameters \PYZob{}
           real theta; // common mean
           real sigma; // common std
         \PYZcb{}
         model \PYZob{}
           for (k in 1:K)
             y[:,k] \PYZti{} normal(theta, sigma);
         \PYZcb{}
         generated quantities \PYZob{}
             matrix[N,K] log\PYZus{}lik;
             for (k in 1:K)
               for (i in 1:N)
                 log\PYZus{}lik[i,k] = normal\PYZus{}lpdf(y[i,k] | theta, sigma);
         \PYZcb{}
         \PYZdq{}\PYZdq{}\PYZdq{}
         
         pool\PYZus{}sm = pystan.StanModel(model\PYZus{}code=pool\PYZus{}model)
         pool\PYZus{}fit = pool\PYZus{}sm.sampling(data=data, iter=2000, chains=4)
         
         hierarchical\PYZus{}model = \PYZdq{}\PYZdq{}\PYZdq{}
         data \PYZob{}
           int\PYZlt{}lower=0\PYZgt{} N;
           int\PYZlt{}lower=0\PYZgt{} K;
           matrix[N,K] y;
         \PYZcb{}
         parameters \PYZob{}
           real theta0; // common theta for each K machine theta
           real\PYZlt{}lower=0\PYZgt{} sigma0; // machine specific sigma
           real theta[K]; // machine specific theta
           real\PYZlt{}lower=0\PYZgt{} sigma; // common std
         \PYZcb{}
         model \PYZob{}
           for (k in 1:K)
             theta[k] \PYZti{} normal(theta0, sigma0);
           for (k in 1:K)
             y[:,k] \PYZti{} normal(theta[k], sigma);
         \PYZcb{}
         generated quantities \PYZob{}
             matrix[N,K] log\PYZus{}lik;
             for (k in 1:K)
               for (i in 1:N)
                 log\PYZus{}lik[i,k] = normal\PYZus{}lpdf(y[i,k] | theta[k], sigma);
         \PYZcb{}
         \PYZdq{}\PYZdq{}\PYZdq{}
         
         hierarchical\PYZus{}sm = pystan.StanModel(model\PYZus{}code=hierarchical\PYZus{}model)
         hierarchical\PYZus{}fit = hierarchical\PYZus{}sm.sampling(data=data, iter=2000, chains=4)
         
         
         def get\PYZus{}separate\PYZus{}result():
             extract\PYZus{}result\PYZus{}from\PYZus{}fit(separate\PYZus{}fit, \PYZsq{}Separate model\PYZsq{})    
             
         def get\PYZus{}pool\PYZus{}result():
             extract\PYZus{}result\PYZus{}from\PYZus{}fit(pool\PYZus{}fit, \PYZsq{}Pool model\PYZsq{})    
             
         def get\PYZus{}hierarchical\PYZus{}result():
             extract\PYZus{}result\PYZus{}from\PYZus{}fit(hierarchical\PYZus{}fit, \PYZsq{}Hierarchical model\PYZsq{})
\end{Verbatim}


    \begin{Verbatim}[commandchars=\\\{\}]
INFO:pystan:COMPILING THE C++ CODE FOR MODEL anon\_model\_e9247cbd108702a6c1e62d7d5ffdd0aa NOW.
INFO:pystan:COMPILING THE C++ CODE FOR MODEL anon\_model\_b3b1d4093fb7ad91b3fc12139a8e74a9 NOW.
INFO:pystan:COMPILING THE C++ CODE FOR MODEL anon\_model\_431a2067d773445b37801edf625af93b NOW.

    \end{Verbatim}


    % Add a bibliography block to the postdoc
    
    
    
    \end{document}
